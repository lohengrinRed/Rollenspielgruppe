\documentclass{book}

% --- Sprachsetup 
\usepackage[T1]{fontenc}    
\usepackage[utf8]{inputenc} 
\usepackage[ngerman]{babel} 

% --- Hyperlinks & PDF bookmarks ---
\usepackage[
  colorlinks=true,   
  linkcolor=blue,    
  urlcolor=blue,     
  bookmarksopen=true 
]{hyperref}

\begin{document}
    \tableofcontents

    \chapter{Im Schatten der Mauer}
    \emph{Geschrieben von Lohengrin:}\\
    Als ich zu mir kamm war ich an einen Phfal gebunden und des Todes nahe in einem Zelt. Umgeben von 3 Gestallten die ich noch nie zuvor gesehen hatte. Die Situation überweltigte mich, doch nachdem der erste Schock sich gelegt hatte, versuchte ich die anderen auf mich Aufmerksam zu machen. Zu meiner Überrachung stellte ich fest, dass die Person neben mir---ich erfuhr später, dass sein Name Urichslaus Birgelbaum ist---ebenfalls wach war. Er erzählte kurz, dass er ein Artist aus Andergast sei. Andergast... da klingelte es bei mir, ich war auf der Reise nach Andergast gewesen um bei einem großen Ritterturnier auszuhelfen. Ich glaube ich wurde von Wegelagerern überracht, ein Schicksal das wohl auch meinen Mitgefangenen zuteil wurde. Der Artist fackelte nicht lange und konnte sich mittels seiner Entfesslungskünste im Handumdrehen von den Fesseln befreien. Wenige Minuten später, waren auch die anderen befreit und bei bewusstsein. Allesamt, mich eingeschlossen, Exoten! Zwei gestallten aus dem fernen Morgenland. Der erste ein Prinz auf der Suche seine Ehre in besagten Tunier auf die Probe zu stellen. Sein Name ist Mharbal Ibn Nadrash. Die Zweite eine Scholarin und Medicus, die auf dem Weg nach Eichhafen war. Sie soll hier schon länger gelebt haben doch hatte noch immer das warmblütige Charisma der Südländerinnen. Ihr name Sephira. So standen wir, unserer Habseeligkeiten beraubt und mit löschrigem Erinnerungen in diesem Zelt, und es geügte nur ein flüchtiger Blick nach Draußen um festzustellen, dass unsere Entführer noch immer vor Ort, und uns vollständig umzingelnd waren.\\
    Nach einer kleinen Vorstellungsrunde war uns allen klar, dass unsere größte Chance zum Überleben es war, besonders tacktvoll und Geschick den Blicken der Feinden zu enkommen, und zu versuchen das Lager leise zu verlassen. Als wir uns waagten eine längere Sekunde damit zu verbringen die Situation außerhalb des Zeltes zu studieren, fielen uns unsere Herzen vom Leib. Die Anlage entpuppte sich als ein, durch eine Mauer befesttigtes, Lager mitsammt Aussichtstürmen, einem Fallgitter als Tor und Wachen die die Mauer patrolieren und im Turm stationiert sind. Auf dem Gelände befindete sich ein lagerfeuer, an dem weitere Entführer saßen und trinkten. Es viel uns eine art kristallernes Instrument auf, mit welchem die Entführer am Lagerfeuer Schelmenspielereien machten. Sephira erklärte, dass dieser Gegenstand ein wertvolles und ihr besonders wichtiges Utensil ist um sehen zu können. Ich beklagte ebenfalls, dass mein wichtiges Schwert, wodurch ich meine Weihe erhielt, fehlte. Als dann auch noch Mharbal erklärte, dass ihm sein Familienerbstück fehlte, war die Situation für uns klar: wir mussten unsere Sachen zurückbekommen um überhaupt eine chance auf Freiheit zu haben. Wir beschlossen uns das Zelt auf der der Mauer zugewanten Seite zu verlassen und in dass nächstgelegene Zeit zu gehen.\\
    Nachdem wir einige Herringe aus dem boden lösen konnten, hoben wir die Zeltplane gerade hochgenug um ins freie zu kriechen. Wir warteten bis wir im toten winkel des Mauerpatrolör waren, und bewegten uns durch den Schatten der Mauer. Wir konnten allesamt in das nächte Zelt gehen und fanden uns in einer Art Lagerraum wieder! Der Raum indem wir waren war Teil eines größerem Zeltes. Durch kleine Vorhänge abgetrennt von unserem Raum schienen zwei weitere Räume zu sein. Aus dem einen hörten wir plotzlich stimmen. Ein würmlicher Mann huldigte scheinbar einem anderen. Der Andere ergriff das Wort und seine stimme war laut und tief. Alleine daran konnte man feststellen, dass dieser Mann ein wahrhaftiger Hühne sein müsse. Er erklärte die Schmeicheleien vom Wurm für ertragslos. Die Ware sei sichergestellt und soll weiterverfrachtet werden. Später erfuhren wir noch, dass wir diese Ware sein. Nachdem wir uns gefasst hatten, und uns bewusst war, dass wir nicht alleine im Zelt waren, erfuhr uns die nächste Hiobsbotschaft; Die Kisten mit der Lagerware waren fest zugenagelt. Ein Aufbrechen der Planken zu laut um unbemerkt zu bleiben, ein Verweilen und langsames Entnagelm wegen unzureichender Ausrüstung zu kostbar unserer begrenzten Zeit. Schweren Herzens schnappten wir uns lose herumliegendes Zeug und machten uns bereit in das nächste Zelt zu gehen.\\
Aus dem nächsten Zelt konnten wir leises Atmen hören. Einer der Wegelagerer schien dort zu schlafen. Da es unsere einzige Chance war an ausrüstung für unseren Schwertarm zu kommen, beschlossen wir uns dem Biedermann zu entledigen. Sephira und Urichslaus hielten den Mann fest während Mharbal ihn erwürgte. Wir hatten Glück und ihm entkroch kein Laut. Der Mann hatte in seinem Zeit neben seinen habseligkeiten noch Waffen: einen Dolch und einen Knüppel. Nachdem wir Mharbal und mich bewaffnet haten, beschlossen wir zurück zu gehen in das große Zelt. Dieses mal konnten wir mit dem Dolch versuchen die Kiste aufzumachen und Fanden dabei den Großteil unserer Ausrüstung, und nebenbei ebenfalls einen Magierstab, den keiner zu bessesn vermochte. Einzig der akrobatenstab von Urichslaus, Mharbals Erbstück, und Sephiras Brille fehlten. Mharbal  erklärte, dass sein Erbstück ein Schwert sei und er es weiter versuchen möchte danach zu suchen. Er waagte sich in den nächtsen Raum zu gehen, obwohl wir nichtmehr die Stimme des vermeintlichen Anführers zu vernehmen vermochten. Der Raum inden er Gang war anscheinend das Schlafgemach und provisorisches Bureau des Anführers. Er durchsuchte den Sekretär und die Truhen die neben dem Bett standen aber von seinem Schwert keine spur. Als er zurück kamm erzählte er mir aber im Geheimen von einem weiteren fund den er gemacht hatte. Die Ware vondem er zuvor geredet hatte, war anscheinend Sephira, welche an eine nicht genauer spezifizierte Organisation weitertransportieren sollte.\\
    Mit dem mulmigen Gefühl, dass wir es mit einer Bande von Menschenhändlern und nicht Gewöhnlichen Wegelagerern zutun hatten, beschlossen wir den anderen Beiden nichts zu den Papieren zu sagen und uns weiter auf die Suche nach unsere letzten Habseligkeiten zu machen. Die Einzigen beiden Zelte die wir nicht aufgesucht hatten waren beide in einer sehr ungünstigen lage. Das eine Zelt war direkt hinter dem Lagerfeuer wo noch die Menschenhändler saßen und trinkten- Das andere befand sich neben unserem Startzelt, doch der Weg dahin führte direkt am Tor und dem Sichfeld des Wachturmes voebei. Zwichen Armee und Meer eingespeert, versuchten wir unsere Glück erneut und beschlossen uns zum Zelt zu gehen, das vor dem Lagerfeuer stand. Es kamm jedoch wie es kommen musste. Ich wurde von der patrolierenden Wache entdeckt und diese ruf sofort den Alarm aus. Ich symbolisierte schnell den anderen, dass sie zurück gehen sollten während ich mich innerlich von ihnen verabschiedete und in das Lagerfeuerzelt lief. Mein Plan war es das Zelt durch ein wunder des Ingarims anzuzünden und durch die verwirung meinen Kammaraden den weg in die Freiheit zu ermöglichen. Doch in diesem hecktischen Moment war mein Glaube nicht gefässtigt genug, kein Funke erschien um das Feuer zu legen.\\
    Die Feinde umstellten schnell mein Zelt indem ich getrennt von meinen Kameraden hockte. Die Drei saßen ebenfalls fest im Großen Zelt und beschlossen ebenfalls ihrem Instink nach zu handeln, Urichslaus lief raus um Sephiras brille vom Lagerfeuer zu hohlen und sah daraufhin, dass das Fallgitter vom Tor im Boden steckte. Er lief in das letzte Zelt um dort in größter Not nach Hoffnung zu suchen. Sephira sah sich der Situation gegenübergestellt, da sie halbblind und ohne kämpferische Fähigkeiten nichts ausrichten konnte. Vom Lagerraum lief sie in das Schlafzimmer um in ihrer größten Not nach Hoffnung zu suchen. Mharbal entschloss sich den Kampf zu suchen, mit einem Knüppel bewaffnet lief er los um mir zu helfen. Ich war schon im Kampf mit den Banditen vom Lagerfeuer verwickelt im engen Zelt, als Mharbal unter der Zeltplane hervorgetreten kamm, sich neben mich stellte, und in meiner größten Not die Hoffnung zu sein nach der ich gesucht hatte.\\
    Urichslaus fand sich in einem Trainingsraum und Bognerwerkstatt wieder. Als er über das Lagergelände lief, zog er die Aufmerksamkeit der Patroullie auf sich die in mit Pfeilen belagerten. Er verschanzte sich im Trainingsraum während Pfeile auf ihn regneten. Aus dem Zelt herraus versuchte er zu verstehen wie das Fallgitter wieder gehoben werden kann doch er sah keinen offentsichlichen Mechanismus, dafür aber die treppe die auf die Mauer und zum Turm führte. Sephira auf der anderen Seite des Lagers, suchte im Raum des Anführers nach irgendwelchen mitteln. Sie fand unter dem Bett des Anführers versteckt Urichslaus' kampfstab. Mharbal und ich bekämpften seite an seite die Menschenhändler. Als der erste zu boden ging schnappte sich Mharbal sein Schwert und machte sich gefasst. Den als der Anführer nachrückte überkamm uns eine Angst vor dem übermächtigen feind. Der Anführer war ein Mann enormer statur, zweieinhalb Schritt groß und vor kraft strozend. An seiner Hüfte Mharbals Schwert, in seiner Hand ein andergaster Bastardschwert. Wir fokusierten uns auf den Anführer und als Mharbal durch eine besonders wuchtige und wahrlich brav getroffenen Hieb den Hühnen zu Boden brachte wendete sich das Blatt für uns. Wir drängte die Unholde aus dem Zelt herraus und beschafften uns Platz zum Atmen.\\
    Sephira lief zu Urichslaus um ihn sein Stab zu bringen. Am Zelt angekommen sahen Sie gerade zu wie Mharbal und ich uns, kräftezerrend,  aus dem Zelt prügelten. Mharbal erschlug einen weiteren Wicht, während Ich den letzten aus das innere des Hofes verfolgte um ihn niederzukämpfen. Das zog aber die Aufmerksammkeit der Schützen auf mich und Pfeile regneten auf mich herab. Als Urichslaus das sah griff er nach den Magierstab, stellte sich gut sichtbar auf den Platz mit dem Stab auf die Schützen zeigent, und rief "Halltet ein oder ich verzaubere euch!".\\
    In ihrer großen verwirrung war ein Schütze für einen kleinen Moment so aus der Fassung gebracht dass er zögerte. Der andere so eingeschüchtert, dass er von der Mauer in den außenliegenden Wald spring---gefolgt von einer sehr ungesund klingenden Landung, die den Mann wohl, so stellte es sich später heraus, das Genick gebrochen hatte. Da ergriffen Urichslaus, Mharbal und ich die chance, und liefen auf die Mauer um den letzten übrig gebliebenden Schützen zu stellen. Es ereignete sich ein Drahtseilakt bei dem wir Drei versuchten den Pfeilen auszuweichen während wir versuchten nicht von der Mauer zu fallen. Schließlich gesah es, dass wir den Mann umzingelten und in einem letzten Manöver von der Mauer stießen. Unsere Entführer waren besiegt.\\
    Wir versorgten unsere Wunden, tauchten uns aus, erkundeten in Ruhe das Lager, verblieben noch eine Nacht im Lager. Wir beschlossen, dass wir, vom Schicksal vereint, noch die letzten Ettapen nach Andergast und Eichhafen hinter uns bringen. Sephira hatte besonderes Interesse unsere Entführer zu Studieren um herrauszufinden, wiso uns das geschehn ist und durch welche Metamorphose der Anführer zu solch gigantischen Ausmass gekommen war. Sie dachte, dass was auch immer hinter unserern Entführern stecke nicht mit natürlichen Mitteln arbeite, und, dass durch das intensive Studium des Leichnams wir mehr über die Organisation herrausfinden könnten. Aufgrund logistischer Probleme beschlossen wir aber den Leichnam in der nächsten Stadt auf unserem Weg zu begrabem. Ebenfalls war uns aufgefallen, dass der Wurm entkommen war und wir wussten, dass es nur eine Frage der Zeit war bis unsere Schicksal wieder zu uns aufschließen würde. Sephira beschloss trozdem nach Eichhafen zu gehen um in ihrem Haus nach dem rechten zu sehen, versicherte uns aber, dass sie zu uns aufschließen würde. Wir anderen Drei machten uns geeint durch den Zauberwald uns erreichten Andergast rechtzeitig zum Ritterturnier.

    \chapter{Des Henkersprinzen Henker}
    \emph{Ein Bericht der Königlichen Stadtwache Andergasts ?}\\
    Der Könglichen Hochheit, dann den denen hoch- und wohlgeboren, edelen, vesten und hochgelehrten, dann respetive hochgeborenen, wohl und hochedelgeborenen Rats- und Kammermitgliedern sei hiermit der Bericht über die Ereignisse der letzten Tage angeraten\dots
    
    \chapter{Der Wolf und die Katze}
    \emph{Aus Marias Tagebuch}\\
    Es ist der Tag der Abreise aus Andergast\dots
\end{document}
