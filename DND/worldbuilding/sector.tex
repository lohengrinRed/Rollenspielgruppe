\documentclass{book}

% --- Sprachsetup 
\usepackage[T1]{fontenc}    
\usepackage[utf8]{inputenc} 
\usepackage[ngerman]{babel} 

% --- Hyperlinks & PDF bookmarks ---
\usepackage[
  colorlinks=true,   
  linkcolor=blue,    
  urlcolor=blue,     
  bookmarksopen=true 
]{hyperref}

\begin{document}
  \chapter{Der Tiefe Himmel}
    Das was die Bewohner vom Ikyan die bekannte Welt nennen hat sich drastisch vor 150 Jahren geändert. Davor waren verteilt auf den 6 Planeten und ihren Monden eine handvoll von keinen Kolonien die von den verschiedenen Völkern bewohnt wurden. Besonders auf den inneren drei felsigen Planeten---Ruyan, Dongo, und Pulvar---gibt es besonders viele und unterschiedliche Kulturen die meistens von riesigen unurbaren landschaften getrennt werden. Einzig in den großen Städten leben die Völker Seite an Seite zusammen und üben sich in Diplomatie, Handel, und kulturellem Austausch. Die drei äußeren Planeten sind Gasriesen, es ist ohne besondere und sehr kostspielige vorbereitung nicht möglich diese Planeten zu besuchen. Jedoch haben diese viele Monde und kleinere Trabanten auf denen sich kleine Raumstationen befinden. Diese orte sind oft sehr isoliert und komunikation zu ihnen ist aufgrund der begrenzten Lichtgeschwindigkeit sehr langsam. Zwischen den verschiedenen Planeten wandeln gelegentlich zivile und kommerzielle raumschiffe, Spelljammer. Die durch Magie betriebenen Konstrukte sind das Herzstück der interplanetaren Reise. Hinter den Gasriesen---Arva, Veryan und Königsende--- befindet sich die Orthale Wolke. Eine riesige Spähre aus Asteroiden die im tiefen Himmel wandeln. Nur sehr wenige von diesen sind metallischer Natur da bei den niedrigen Temperaturen sogar Gase anfangen zu gefrieren. Die sporadisch vorkommenden, besonders wertvolle Gase haben die Zwerge dazu gebracht vor vielen Jahrhunderten in der Orthalen Wolke weltraumstationen zu bauen, sodass jetzt in Ikyan alle Zwerge in dieser lebensfeindlichen Region leben. Selten kommen Händler in zu den Planeten doch sie verweilen nicht lange.\\
    Vor 150 Jahren jedoch erschien aber ohne vorwarnung ein großes Schlachtschiff im System. Die Winterrose stammt von der großen Republik, ein Staatenbund vor Tausend Jahren Ikyan besiedelt und aufgebaut hatte doch dann nach einem kataklysmischen Ereignis die Verbindung zu dem Sektor verloren hatten. Nach einem wenige Wochen andauernden Krieges, war die gesamte Kontrolle über Ikyan in die Hand der großen Republik übergegangen. Da durch den schlagartigen Anstieg an Material- und Arbeitszeit-Abgaben wird durch eine starke Truppenpräsenz die öffentliche ordnung\\
    Nach der Herstellung der himmlischen Ordnung investierte die große Republik eine große Menge an Arbeitskraft zum Bau eines Weltentorhafens hinter der Helipause. Die dort stationierten Portale werden von Schiffen der großen Republik benutzt um den Transport in die anderen Sektoren zu ermöglichen. Den normalen Bewohnern ist es aber durch die extrem hohen Strahlungen nicht möglich mit gewöhnlichen Spelljammern die Heliopause zu verlassen.
  \chapter{Dongo}
    Ein wahres Paradies für Entdecker ist Dongo. Der Wüstenplanet ist am wenigsten erkundet und seine Landstriche sind von gigantischen Wehen geprägt. Die wellenartige Form und die ständige Bewegung des Sandes durch die tiefen Winde brachten der Wüste den Namen Dongo Dilizinga. Ohne jegliche Plattentektonik ist der Planet arm an natürlichen Rohstoffen sodass sich große Besiedlung nie gelohnt hat. Die sich bewegenden Wehen offenbaren jedoch gelegentlich verschollene Komplexe die gelegentlich wahre Schätze beinhalten. Abenteurer und Entdecker reiten auf Sandschiffen über die windenden Sande.\\
    Der Wasserarme Planet schien in der Vergangenheit einen eigenen Mond gehabt haben denn ein dünner Ring aus Gestein ziert den golden rot schimmernden Planeten. Die kleinen brocken haben aufgrund ihrer fehlenden Atmosphäre eine sehr geringe Temperatur. An Tagen der orbitale Resonanz, regnen diese brocken auf den Planeten und das kühlende Gestein mit Wassereinschlüssen bringt Segensreiche ernten.\\
    Es liegt an den Leuten Dongos ob sie ihren Tag in den örtlichen Teehäusern, Badeanstalten, auf den Plantagen der Oasen, in den windenden Sänden oder auf den kühlen eisringen verbringen wollen. In den Sand von Dongo kann jede Geschichte geschrieben werden.
  \chapter{Ruyan}
    Ruyan ist ein sehr kleiner felsiger Planet. Die dünne Atmosphäre macht längere unterfangen sehr anstrengend weswegen das wirtschaftliche treiben auf Ruyan sehr langsam verläuft. Doch seit nicht überacht wenn die Elfen der Ruya schlagartig ihr gemüt ändern und heißblütig über Liebe, Politik oder Philosophie streiten. Man sagt das sie das licht der Sonne soweit in sich aufgenommen haben das alleine ihre Präsenz die Stimmung in jedem Raum aufheizt. Ihre reiche Kultur voll von Musik, Tanz, und Kunst macht sie zu den besten Handwerkern Ikyan---falls man sie dazu überredet bekommt ein solches unterfangen anzufangen. Eine Muse ist natürlich Pflicht für ein ruyanisches Meisterwerk\dots\\
   \chapter{Pulvar}
    Pulvar ist ein sehr waldiger Planet mit großen türkisen Ozeanen. Hier leben alle möglichen Völker (von den Zwergen abgesehen) da es reichlich Essen und Platz gibt. Auf Pulvar befinden sich auch die größten Städte Ikyans mit ihren eigenen regionalen Besonderheiten als Schmelzziegel der Völker. Aufgrund dieser hochkultur hat hier auch große Republik ihren lokalen Regierungssitz aufgebaut. Der große Reichtum der dadurch nach Pulvar gekommen ist hat zieht Siedler aus ganz Ikyan an\dots\\
  \chapter{Die Gasriesen}
    Die vielen Monde der Gasriesen zu beschreiben ist ein Unterfangen das viel Zeit bräuchte da sie alle sehr unterschiedlich in ihrer Natur sind. Es gibt Trabanten so so klein wie Asteroiden und so groß wie Planeten. Manche haben ihre eigene Atmosphäre andere nicht. Manche sind gigantische wasserwelten andere öde wüsten. Arm, reich, warm, kalt, vulkanisch, still, was man sich vorstellen kann und sogar unglaubliche orte kann man dort finden. Sie alle eint der Mut der Siedler die diese zum Teil lebensbedrohlichen landschaften sahen und sagten "Hier möchte ich mein Leben aufbauen"\dots
  \chapter{Außerhalb Ikyans}
    Ikyan ist der vierte der Sektroen des großen Imperiums. Diese jedes mindestens so groß wie Ikyan mit einer fülle unterschiedlicher Völkern. Doch auch die vier Sektoren sind nicht alles was es in der Welt gibt, weit Außerhalb der Sternensysteme befindet sich eine Welt die es zu entdecken gibt, mit seinen eigenen phantastischen nebeln, Sternenhaufen, kosmische leerräume und weiteren großskaligen Strukturen des Universums. Wesen wie Dämonen, Feen, Teufeln, Engeln, Automaten und viele mehr sollen diese orte bewohnen. Wege dorthin werden wahrscheinlich nur durch Koorperation mit diesen phantastischen Wesen entdeckt werden. Wer weiß was in der Zukunft auf die große Republik zukommt\dots

\chapter{Von den Prinzipien die die Welt bestimmen}
Es gibt bestimmte Prinzipien oder Mechanismen die die Realität definieren. Diese Prinzipien haben unabhängig voneinander die verschiedensten Völker entdeckt und zu beschreiben versucht. Hier wird nur ein kleiner einblick in ihre Wesen gewährt aber am Ende ist es der Glaube der aus Prinzipien einen Gott macht. Und so wie jeder ein anderes Bild dieser Prinzipien hat so drücken sich die Priester und Gotteskrieger ihren Glauben auf unterschiedliche Art und Weise aus.

\section{Entropie}
Stetig nagt der Zahn der Zeit. Metall verrostet, Laub wird zu Humus, Berge weichen Flüssen, und Wesen Sterben. Auch wenn für einen Moment die Ordnung wächst, strebt das System immerzu an durcheinander zu geraten. Das ist das Wesen der Entropie, es beinhaltet Zerfall und Wandel. Die Jünger der Entropie verschreiben sich zum Beispiel dem Tod von Wesen, führen Historie über die Welt, oder befassen sich mit der Mehrung und Wahrung der natürlichen Zerstörung. Eine Kriegerin der Entropie würde sich zum Beispiel in den Kampf gegen unnatürlichen Untot verschreiben. Die genauen Eide sind sehr persönlich und unterscheiden sich Teilweise sehr. So verschieden sind die Eide gleich der Art der Entropie.

\section{Synthese}
Man sagt das erste Leben auf der Welt entstand im Zufall. Moleküle ordneten sich zufällig an im Inneren eines Vulkan, Unterwasser und abgeschirmt von Strahlung. Doch von Zufall kann es nicht sein: Falls die Vorausetzungen perfekt vorhanden sind wird es immer dazu kommen. Das ist die Natur der Synthese, es beinhaltet Ordnung, Wachstum, und Verknüpfung. Die Jünger der Synthese üben sich zum Beispiel in der Schaffung neuer Verbindungen, Kunstwerken, oder Ehen. Sie sind manchmal als Heiler tätig oder unterstützen große Bauvorhaben. Manche Jünger verschreiben sich sogar der Unterstützung der Ausgebeuteten. Eine Kriegerin der Synthese würde sich zum Beispiel in den Kampf gegen Menschenhändler verschreiben. Auch hier gibt es sehr unterschiedliche Missionen. So vielseitig sind die Eide gleich der Art der Synthese.

\section{Potential}
Situationen in denen der Ausgang ungewiss ist begegnen uns jeden Tag: Ein Felsen steht auf einen Berg, im Tal links ein Dorf, im Tal rechts ein Fluss, wo wird der Felsen hinrollen? Für einen mag es ungewiss sein aber der Gelehrte mag wissen wie der Wind sich drehen wird. So ist das Wesen des Potentiales, es beinhaltet Möglichkeiten, Zufall, und Wahrscheinlichkeiten. Ein Jünger des Potential mag sich dem Sammeln von Wissen zu verschreiben, sucht Ordnung in zufällig erscheinenden Prozessen, oder versucht die das Schicksal vorauszusagen oder zu manipulieren. Eine Kriegerin des Potential würde sich zum Beispiel in den Kampf gegen Demagogen verschreiben. Jeder Auftrag ist mit einem persönliches Schicksal verbunden. So ungewiss sind die Eide gleich der Art des Potential.

\section{Resonanz}
Wissenschaftler haben Experimente konstruiert die ihr Ergebnis ändert wenn es Beobachtet wird oder nicht. Andersherum muss also auch die Welt einen Effekt auf den Geist haben. Menschen sagen der Glaube kann Berge bewegen doch gehen sie selber zum Berg um ihren Glauben zu bewegen. Genau dort wird die Verbindung zwischen Geist und Welt am klarsten, das ist das Wesen der Resonanz. Jünger der Resonanz verschreiben sich zum Beispiel in Mediation zwischen Kulturen und der Natur, Besänftigen Geister von Lebenden und Toten, oder begeben sich auf die mystische suche nach dem Geist der Welt. Eine Kriegerin der Resonanz würde sich zum Beispiel in den Kampf gegen einen Naturschänder verschreiben. Die Gelübde sind stark abhängig von der Natur die sie umgeben. So multiplex sind die Eide gleich der Art der Resonanz.

\section{Gravitation}
Es scheint eine Kraft zu geben die Materie zu einander Anzieht. Aus Staubwolken werden Planeten, die Sternensysteme wirren um ein Zentrum, die Zentren bilden lange Spinnennetz-förmige Ketten, und Menschen begegnen sich immer zweimal im Leben. So ist das Wesen der Gravitation, es beinhaltet Zusammenhalt und Treue. Jünger der Gravitation befassen sich zum Beispiel mit Parteien, Ländern und Unionen. Sie fungieren als Diplomaten oder Politiker aber können auch sich dem Mystizismus der Weltentstehung oder Kosmologie verschreiben. Eine Kriegerin der Gravitation würde sich zum Beispiel in den Kampf gegen Eidbrecher verschreiben. Jedes Versprechen ist ein heiliger und bilateraler Akt. So universell ist der Eid gleich der Art der Gravitation.

\section{Chaos}
Log eintrag: \emph{Mein Schiff segelt schon seit Monaten durch die tiefe dunkele Leere. Kein Sternenlicht das mir den weg weist. Die Detektoren zeigen den Ausendruck an: 1 Atom pro kubikmeter. Hier bin ich ganz allein. Mein Schiff fängt langsam aus Auszugasen. Die Atome der Hülle wollen unbedingt die Leere draußen Füllen. Das Metall wird zu Gas. Nicht nur mein Schiff wird zerrissen. Ich fühle wie {\bf es} an meinem verstand zerrt. Ich weis, dass {\bf es} hier ist. Umso tiefer ich in den leeren Raum fahre umso lauter wird das Rauchen meines Radios. Hier sollte nichts sein. Kein Teilchen da um wärmestrahlung abzugeben. Der weite ewige Raum. Ich glaube {\bf es} hat mich nicht einmal bemerkt. Doch ich habe {\bf es} bemerkt. Ist {\bf es} das was die Hexenmeister Chaos nennen, Tohuwabohu\dots}


\end{document}
